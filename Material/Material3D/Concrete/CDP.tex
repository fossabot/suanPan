\documentclass[10pt,fleqn,3p]{elsarticle}
\usepackage{amsmath,amsfonts,amssymb,mathpazo,indentfirst}
\newcommand*{\md}[1]{\mathrm{d}#1}
\newcommand*{\mT}{\mathrm{T}}
\newcommand*{\tr}[1]{\mathrm{tr}#1}
\newcommand*{\ddfrac}[2]{\dfrac{\md#1}{\md#2}}
\newcommand*{\pfrac}[2]{\dfrac{\partial#1}{\partial#2}}
\title{CDP Model}\date{}\author{tlc}
\begin{document}\pagestyle{empty}
This is a brief summary of the CDP model implemented in \texttt{suanPan}. For detailed theories, readers are recommended to check the original literature.
\section{Yield Function}
The yield function is defined as
\begin{gather}
F=\alpha{}I_1+\sqrt{\dfrac{3}{2}}|s|+\beta\left<\sigma_1\right>-\left(1-\alpha\right)c_c,
\end{gather}
with $c_c=-\bar{f}_c$ and $\beta=\dfrac{\bar{f}_c}{\bar{f}_t}(\alpha-1)-(\alpha+1)$.
\section{Flow Rule}
The flow potential is chosen to be
\begin{gather}
G=\sqrt{2J_2}+\alpha_pI_1=|s|+\alpha_p\tr{\sigma}.
\end{gather}
The flow rule is accordingly defined as
\begin{gather}
\Delta\varepsilon^p=\Delta\lambda\pfrac{G}{\sigma}=\Delta\lambda\left(\dfrac{s}{|s|}+\alpha_pI\right),
\end{gather}
so that
\begin{gather}
\Delta\varepsilon^p_d=\Delta\lambda\dfrac{s}{|s|},\quad\Delta\varepsilon^p_v=3\alpha_p\Delta\lambda.
\end{gather}

Noting that $s=s^{tr}-2G\Delta\varepsilon^p_d=s^{tr}-2G\Delta\lambda\dfrac{s}{|s|}$ and $p=p^{tr}-K\Delta\varepsilon^p_v=p^{tr}-3K\alpha_p\Delta\lambda$, equivalently,
\begin{gather}
|s|=|s^{tr}|-2G\Delta\lambda,\quad{}I_1=I_1^{tr}-9K\alpha_p\Delta\lambda.
\end{gather}
Furthermore,
\begin{gather}
\dfrac{s}{|s|}=\dfrac{s^{tr}}{|s^{tr}|}\equiv{}n,
\end{gather}
so that
\begin{gather}
\Delta\varepsilon^p=\Delta\lambda\left(n+\alpha_pI\right).
\end{gather}
The flow direction is fixed for each sub-step.
\section{Damage Evolution}
Damage parameters shall satisfy the following expression.
\begin{gather}
\kappa-\kappa_n-H\Delta\lambda=0.
\end{gather}
For tension and compression, two separate scalar equations can be written as
\begin{gather}
\kappa^t_n+r\dfrac{f_t}{g_t}(n_1+\alpha_p)\Delta\lambda-\kappa^t=0,\\
\kappa^c_n+(1-r)\dfrac{f_c}{g_c}(n_3+\alpha_p)\Delta\lambda-\kappa^c=0.
\end{gather}

The backbone curves are related to the damage parameter $\kappa_\aleph$.
\begin{gather*}
f_\aleph=f_{\aleph,0}\sqrt{\phi_\aleph}\Phi_\aleph,
\end{gather*}
with
\begin{gather*}
\phi_\aleph=1+a_\aleph\left(2+a_\aleph\right)\kappa_\aleph,\quad
\Phi_\aleph=\dfrac{1+a_\aleph-\sqrt{\phi_\aleph}}{a_\aleph}.
\end{gather*}

The effective counterpart is defined as
\begin{gather*}
\bar{f}_\aleph=\dfrac{f_\aleph}{1-d}=f_{\aleph,0}\sqrt{\phi_\aleph}\Phi_\aleph^{1-c_\aleph/b_\aleph},
\end{gather*}
with
\begin{gather*}
d=1-\Phi_\aleph^{c_\aleph/b_\aleph}.
\end{gather*}

The corresponding derivatives are
\begin{gather*}
\ddfrac{d}{\kappa_\aleph}=\dfrac{c_\aleph}{b_\aleph}\dfrac{a_\aleph+2}{2\sqrt{\phi_\aleph}}\Phi_\aleph^{c_\aleph/b_\aleph-1},\quad
\ddfrac{f_\aleph}{\kappa_\aleph}=f_{\aleph,0}\dfrac{a_\aleph+2}{2\sqrt{\phi_\aleph}}\left(a_\aleph-2\sqrt{\phi_\aleph}+1\right),\\
\ddfrac{\bar{f}_\aleph}{\kappa_\aleph}=f_{\aleph,0}\dfrac{a_\aleph+2}{2\sqrt{\phi_\aleph}}\dfrac{\left(a_\aleph+1+\left(\dfrac{c_\aleph}{b_\aleph}-2\right)\sqrt{\phi_\aleph}\right)}{\Phi_\aleph^{c_\aleph/b_\aleph}}.
\end{gather*}
\section{Residual}
The yield function and the damage evolutions are three local equations shall be satisfied.
\begin{gather}
R\left\{
\begin{array}{l}
\alpha{}I_1^{tr}-9K\alpha\alpha_p\Delta\lambda+\sqrt{\dfrac{3}{2}}\left(|s^{tr}|-2G\Delta\lambda\right)+\beta\left<\sigma_1^{tr}-\Delta\lambda\left(2Gn_1+3K\alpha_p\right)\right>+\left(1-\alpha\right)\bar{f}_c=0,\\
\kappa^t_n+r\dfrac{f_t}{g_t}(n_1+\alpha_p)\Delta\lambda-\kappa^t=0,\\
\kappa^c_n+(1-r)\dfrac{f_c}{g_c}(n_3+\alpha_p)\Delta\lambda-\kappa^c=0.
\end{array}
\right.
\end{gather}

By choosing $x=\begin{bmatrix}
\Delta\lambda&\kappa_t&\kappa_c
\end{bmatrix}^\mT$ as independent variables and assuming $n=\dfrac{s^{tr}}{|s^{tr}|}$ that is a function of $\varepsilon^{tr}$ only thus does not contain $\Delta\lambda$, the Jacobian can be computed as
\begin{gather*}
J=\begin{bmatrix}
-9K\alpha\alpha_p-\sqrt{6}G-\beta\left(2Gn_1+3K\alpha_p\right)H(\sigma_1)&\left\langle\sigma_1\right\rangle\pfrac{\beta}{\kappa_t}&(1-\alpha)\bar{f}_c'+\left\langle\sigma_1\right\rangle\pfrac{\beta}{\kappa_c}\\[4mm]
f_t\dfrac{n_1+\alpha_p}{g_t}(r+\Delta\lambda\pfrac{r}{\Delta\lambda})&r\Delta\lambda\dfrac{n_1+\alpha_p}{g_t}f_t'-1&\cdot\\[4mm]
f_c\dfrac{n_3+\alpha_p}{g_c}(1-r-\Delta\lambda\pfrac{r}{\Delta\lambda})&\cdot&(1-r)\Delta\lambda\dfrac{n_3+\alpha_p}{g_c}f_c'-1
\end{bmatrix},
\end{gather*}
where
\begin{gather*}
\pfrac{\beta}{\kappa_t}=(1-\alpha)\dfrac{\bar{f}_c}{\bar{f}_t^2}\bar{f}_t',\qquad\pfrac{\beta}{\kappa_c}=(\alpha-1)\dfrac{1}{\bar{f}_t}\bar{f}_c'.
\end{gather*}
In explicit form, if $\sigma_1>0$,
\begin{gather}
J=\begin{bmatrix}
-9K\alpha\alpha_p-\sqrt{6}G-\beta\left(2Gn_1+3K\alpha_p\right)&(1-\alpha)\dfrac{\bar{f}_c\sigma_1}{\bar{f}_t^2}\bar{f}_t'&(1-\alpha)(1-\dfrac{\sigma_1}{\bar{f}_t})\bar{f}_c'\\[4mm]
f_t\dfrac{n_1+\alpha_p}{g_t}(r+\Delta\lambda\pfrac{r}{\Delta\lambda})&r\Delta\lambda\dfrac{n_1+\alpha_p}{g_t}f_t'-1&\cdot\\[4mm]
f_c\dfrac{n_3+\alpha_p}{g_c}(1-r-\Delta\lambda\pfrac{r}{\Delta\lambda})&\cdot&(1-r)\Delta\lambda\dfrac{n_3+\alpha_p}{g_c}f_c'-1
\end{bmatrix},
\end{gather}
otherwise,
\begin{gather}
J=\begin{bmatrix}
-9K\alpha\alpha_p-\sqrt{6}G&\cdot&(1-\alpha)\bar{f}_c'\\[4mm]
f_t\dfrac{n_1+\alpha_p}{g_t}(r+\Delta\lambda\pfrac{r}{\Delta\lambda})&r\Delta\lambda\dfrac{n_1+\alpha_p}{g_t}f_t'-1&\cdot\\[4mm]
f_c\dfrac{n_3+\alpha_p}{g_c}(1-r-\Delta\lambda\pfrac{r}{\Delta\lambda})&\cdot&(1-r)\Delta\lambda\dfrac{n_3+\alpha_p}{g_c}f_c'-1
\end{bmatrix}.
\end{gather}
\section{Consistent Tangent Stiffness}
Taking derivatives with regard to trial strain of the residual equations gives
\begin{gather}
\pfrac{R}{\varepsilon^{tr}}=\left\{
\begin{array}{l}
3K\alpha{}I+\sqrt{6}Gn+\beta\ddfrac{\hat\sigma}{\bar\sigma}_t\pfrac{\bar\sigma}{\varepsilon^{tr}}H(\sigma),\\
\dfrac{f_t}{g_t}\Delta\lambda\left(r\ddfrac{\hat\sigma}{\bar\sigma}_t\ddfrac{n}{e}+\left(n_1+\alpha_p\right)r'\ddfrac{\hat\sigma}{\bar\sigma}\pfrac{\bar\sigma}{\varepsilon^{tr}}\right),\\
\dfrac{f_c}{g_c}\Delta\lambda\left(\left(1-r\right)\ddfrac{\hat\sigma}{\bar\sigma}_c\ddfrac{n}{e}-\left(n_3+\alpha_p\right)r'\ddfrac{\hat\sigma}{\bar\sigma}\pfrac{\bar\sigma}{\varepsilon^{tr}}\right).
\end{array}
\right.
\end{gather}
so that
\begin{gather}
\ddfrac{x}{\varepsilon^{tr}}=\begin{bmatrix}
\ddfrac{\Delta\lambda}{\varepsilon^{tr}}\\[3mm]
\ddfrac{\kappa_t}{\varepsilon^{tr}}\\[3mm]
\ddfrac{\kappa_c}{\varepsilon^{tr}}
\end{bmatrix}=-\pfrac{R}{x}^{-1}\pfrac{R}{\varepsilon^{tr}}.
\end{gather}
In above equations, $\ddfrac{\hat\sigma}{\bar\sigma}$ is the transformation matrix between principal stress and nominal stress, which is a function of eigen vectors and does not change with any other variables. So it can be treated as a constant transform matrix.

The stress update is computed as follows.
\begin{gather}
\sigma=(1-d_c)(1-sd_t)\bar{\sigma}.
\end{gather}
The tangent stiffness of which is
\begin{gather}
\ddfrac{\sigma}{\varepsilon^{tr}}=(1-d_c)(1-sd_t)\ddfrac{\bar{\sigma}}{\varepsilon^{tr}}+\bar{\sigma}\otimes\ddfrac{(1-d_c)(1-sd_t)}{\varepsilon^{tr}}.
\end{gather}

The effective stress $\bar{\sigma}$ only depends on $\varepsilon^{tr}$ and $\Delta\lambda$.
\begin{gather}
\begin{split}
\ddfrac{\bar{\sigma}}{\varepsilon^{tr}}&=\ddfrac{\left(s^{tr}-2G\Delta\lambda\dfrac{s^{tr}}{|s^{tr}|}+(p^{tr}-3K\alpha_p\Delta\lambda)I\right)}{\varepsilon^{tr}}\\
&=\ddfrac{\left(2G\varepsilon_d^{tr}-2G\Delta\lambda\dfrac{\varepsilon_d^{tr}}{|\varepsilon_d^{tr}|}+\left(K\varepsilon_v^{tr}-3K\alpha_p\Delta\lambda\right)I\right)}{\varepsilon^{tr}}\\
&=2GI_d-\dfrac{4G^2\Delta\lambda}{|s^{tr}|}\left(I_d-n\otimes{}n\right)+KI\otimes{}I-\left(2Gn+3K\alpha_pI\right)\otimes\ddfrac{\Delta\lambda}{\varepsilon^{tr}}\\
&=D^e-\dfrac{4G^2\Delta\lambda}{|s^{tr}|}\left(I_d-n\otimes{}n\right)-\left(2Gn+3K\alpha_pI\right)\otimes\ddfrac{\Delta\lambda}{\varepsilon^{tr}}
\end{split},
\end{gather}
in which
\begin{gather}
\pfrac{\bar{\sigma}}{\varepsilon^{tr}}=D^e-\dfrac{4G^2\Delta\lambda}{|s^{tr}|}\left(I_d-n\otimes{}n\right).
\end{gather}

The damage factor can be expressed as
\begin{gather}
\begin{split}
\ddfrac{(1-d_c)(1-sd_t)}{\varepsilon^{tr}}&=\ddfrac{\left(1-sd_t-d_c+sd_td_c\right)}{\varepsilon^{tr}}=\ddfrac{\left(sd_td_c\right)}{\varepsilon^{tr}}-\ddfrac{\left(sd_t\right)}{\varepsilon^{tr}}-\ddfrac{d_c}{\varepsilon^{tr}}\\
&=d_td_c\ddfrac{s}{\varepsilon^{tr}}+sd_c\ddfrac{d_t}{\varepsilon^{tr}}+sd_t\ddfrac{d_c}{\varepsilon^{tr}}-d_t\ddfrac{s}{\varepsilon^{tr}}-s\ddfrac{d_t}{\varepsilon^{tr}}-\ddfrac{d_c}{\varepsilon^{tr}}\\
&=d_t(d_c-1)\ddfrac{s}{\varepsilon^{tr}}+s(d_c-1)\ddfrac{d_t}{\varepsilon^{tr}}+(sd_t-1)\ddfrac{d_c}{\varepsilon^{tr}}\\
&=d_t(d_c-1)(1-s_0)\ddfrac{r}{\varepsilon^{tr}}+s(d_c-1)d_t'\ddfrac{\kappa_t}{\varepsilon^{tr}}+(sd_t-1)d_c'\ddfrac{\kappa_c}{\varepsilon^{tr}}
\end{split},
\end{gather}
where
\begin{gather}
\ddfrac{r}{\varepsilon^{tr}}=\ddfrac{r}{\hat\sigma}\ddfrac{\hat\sigma}{\bar\sigma}\ddfrac{\bar\sigma}{\varepsilon^{tr}}.
\end{gather}
\end{document}
